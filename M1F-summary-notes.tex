\documentclass{article}
\title{M1F Summary Notes (JMC Year 1, 2017/2018 syllabus)}
\date{}
\author{Fawaz Shah}

% Packages for adding hyperlinks to table of contents
\usepackage{color}   %May be necessary if you want to color links
\usepackage{hyperref}
\hypersetup{
    colorlinks=true, %set true if you want colored links
    linktoc=all,     %set to all if you want both sections and subsections linked
    linkcolor=black,  %choose some color if you want links to stand out
}

%package that allows aligned equations
\usepackage{amsmath}

%package that allows notation for extra mathematical symbols
\usepackage{amssymb}

%new commands for popular sets for ease of use
\newcommand{\R}{\mathbb{R}}
\newcommand{\N}{\mathbb{N}}
\newcommand{\Z}{\mathbb{Z}}

% renaming command to writing vectors in bold notation
\renewcommand{\vec}[1]{\mathbf{#1}}

%package for managing images
\usepackage{graphicx}
\graphicspath{ {img/} }

%package for managing hyperlinks
\usepackage{hyperref}

%package for added blue colored boxes
\usepackage{tcolorbox}

%package that allowed removing indent from enumerate environment
\usepackage{enumitem}

\begin{document}
\large
\maketitle
\begin{center}
UNDER CONSTRUCTION
\end{center}
\noindent This document contains a list of definitions and a list of theorems.
\\\\
Note that the exam will probably require you to PROVE some of these theorems, so you should refer back to the original notes for the proofs.
\\\\
Boxes cover content in more detail.
\tableofcontents
\newpage

\section{Definitions}
\paragraph{Arbitrary union}
We can define the arbitrary union
\begin{equation}
\bigcup_{i \in I} X_{i}
\end{equation}
to be the union of all sets $ X_{i} $.
\paragraph{Arbitrary intersection}
We can define the arbitrary intersection
\begin{equation}
\bigcap_{i \in I} X_{i}
\end{equation}
to be the intersection of all sets $ X_{i} $.
\paragraph{Modulus (complex number)}
\paragraph{Argument (complex number)}

\section{Theorems}
\subsection{Logic and Sets}
\textit{Contrapositive law}
\begin{equation}
(P \Rightarrow Q) \Leftrightarrow (\neg Q \Rightarrow \neg P)
\end{equation}
\\
To negate any logical statements, we can flip the $ \forall $ and $ \exists $ signs and negate the predicate (the mathematical statement at the end).
\subsection{Complex numbers}
Complex numbers can be added, subtracted, multiplied and divided.
\\\\
For any complex number $ z $,
\begin{equation}
z \overline{z} = |z| ^{2}
\end{equation}
\\
For any complex numbers $ z_{1}, z_{2} $:
\begin{align}
\overline{z_{1} + z_{2}} & = \overline{z_{1}} + \overline{z_{2}} \\
\overline{z_{1} - z_{2}} & = \overline{z_{1}} - \overline{z_{2}} \\
\overline{z_{1} z_{2}} & = \overline{z_{1}} \cdot \overline{z_{2}} \\
\overline{(\frac{z_{1}}{z_{2}})} & = \frac{\overline{(z_{1})}}{\overline{(z_{2})}} \\
\overline{(z_{1})^{n}} & = (\overline{z_{1}})^{n}
\end{align}
The inverse of a complex number $ z $ is such that:
\begin{equation}
z z^{-1} = 1
\end{equation}
\\
Any complex number $ z $ can be represented in exponential form:
\begin{equation}
z = r e^{i \theta}
\end{equation}
where $ r $ is the modulus of $ z $ and $ \theta $ is the argument.
\\\\
\textit{Euler's formula}
\begin{equation}
e^{i \theta} = \cos(\theta) + i \sin(\theta)
\end{equation}
This can be proven using power series.
\\\\
\textit{De Moivre's Theorem}
\\
For any complex number $ z $:
\begin{equation}
z^{n} = r^{n} (\cos(n \theta) + i \sin(n \theta))
\end{equation}
This can be proven by induction or using Euler's formula.
\\\\
\textit{Fundamental Theorem of Algebra}
\\
Any polynomial of degree $ n $ has $ n $ roots in the complex plane.
\\\\
\textit{Roots of unity}
\\
Any root of unity (i.e. a solution to $ z^{n} = 1 $) can be expressed in the form:
\begin{equation}
z = e^{\frac{2 \pi k i}{n}}, \quad k \in \Z_{\geq 0}
\end{equation}

\subsection{Number theory}
\textit{Weak induction}
\\
If we take $ P $ as a predicate:
\begin{equation}
P(0) \land (P(k) \Rightarrow P(k+1)) \Rightarrow P(n) \quad (\forall n)
\end{equation}
Note that we do not have to start at 0:
\begin{equation}
P(m) \land (P(k) \Rightarrow P(k+1)) \Rightarrow P(n) \quad (\forall n \geq m)
\end{equation}
\\
\textit{Strong induction}
\\
If we take $ P $ as a predicate:
\begin{equation}
P(0) \land ((\forall j \leq k \ P(j)) \Rightarrow P(k+1)) \Rightarrow P(n) \quad (\forall n)
\end{equation}
Note that we do not have to start at 0:
\begin{equation}
P(m) \land ((\forall j \leq k \ P(j)) \Rightarrow P(k+1)) \Rightarrow P(n) \quad (\forall n \geq m)
\end{equation}
\begin{tcolorbox}
We must prove $ P(k) \Rightarrow P(k+1) $ for weak induction.
\\
We must prove $ (\forall j \leq k \ P(j)) \Rightarrow P(k+1) $ for strong induction.
\\\\
Strong induction is mathematically equivalent to weak induction.
\end{tcolorbox}
\noindent
\\
\textit{Completeness Axiom}
\\
A set $ S $ has a least upper bound (sup) iff:
\begin{itemize}
\item $ S $ is non-empty
\item $ S $ is bounded above
\end{itemize}
Corresponding statement for greatest lower bound (inf).
\\\\
Every real number has a decimal expansion.
\\\\
A number is rational $ \Leftrightarrow $ it has a periodic decimal expansion.
\\\\
For any $ x \in \Z_{\geq 0} $, $ x \ | \ a \textrm{ and } x \ | \ b \Rightarrow x \ | \ \gcd(a, b) $.
\\\\
\textit{Euclid's Algorithm}
\\
To find the $ gcd $ of $ a $ and $ b $ where $ a > b $. Write:
\begin{align}
a & = q_{1} b + r_{1} \\
b & = q_{2} r_{1} + r_{2} \\
r_{1} & = q_{3} r_{2} + r_{3} \\
& \vdots \\
r_{n-1} & = q_{n} r_{n-1} + r_{n}
\end{align}
Continue until $ r_{n} = 0 $, return $ r_{n-1} $.
\\
\begin{tcolorbox}
In Haskell notation:
\\\\
$ \gcd(a, 0) = 0 $
\\
$ \gcd(a, b) = \gcd(b, a \textrm{ mod } b) $
\end{tcolorbox}
\noindent
\\
\textit{Bezout's Theorem}
\begin{equation} \label{bezout}
\gcd(a, b) = \lambda a + \mu b \quad (\textrm{for some } \lambda, \mu \in \R)
\end{equation}
\begin{tcolorbox}
We can write $ a $ and $ b $ as:
\begin{align}
a & = \alpha \gcd(a, b) \\
b & = \beta \gcd(a, b)
\end{align}
In general the solution to equation \ref{bezout} is given by:
\begin{equation}
\gcd(a, b) = (\lambda + \beta n) a + (\mu - \alpha n) b
\end{equation}
noting that the extra terms will always cancel out. So we have a set of solutions ($ \lambda_{n}, \mu_{n} $), where:
\begin{equation}
\lambda_{n} = \lambda + \beta n, \quad \mu_{n} = \mu - \alpha n
\end{equation}
\end{tcolorbox}
\noindent
\\
Every integer larger than $ 1 $ can be written as a product of primes.
\\
(use strong induction to prove)
\\\\
\textit{Fundamental Theorem of Arithmetic}
\\
Every integer larger than $ 1 $ can be written UNIQUELY as a product of primes. (proof not needed)
\subsection{Equivalence relations and functions}
\subsection{Combinatorics}

\end{document}