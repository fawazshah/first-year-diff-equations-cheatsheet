%using document class from KOMA-Script
\documentclass{scrartcl}
\title{M1F Summary Notes}
\subtitle{JMC Year 1, 2017/2018 syllabus}
\date{}
\author{Fawaz Shah}

% Packages for adding hyperlinks to table of contents
\usepackage{color}   %May be necessary if you want to color links
\usepackage{hyperref}
\hypersetup{
    colorlinks=true, %set true if you want colored links
    linktoc=all,     %set to all if you want both sections and subsections linked
    linkcolor=black,  %choose some color if you want links to stand out
}

%package that allows aligned equations
\usepackage{amsmath}

%package that allows notation for extra mathematical symbols
\usepackage{amssymb}

%new commands for popular sets for ease of use
\newcommand{\R}{\mathbb{R}}
\newcommand{\N}{\mathbb{N}}
\newcommand{\Z}{\mathbb{Z}}
\newcommand{\Q}{\mathbb{Q}}
\newcommand{\C}{\mathbb{C}}

% renaming command to writing vectors in bold notation
\renewcommand{\vec}[1]{\mathbf{#1}}

%package for managing images
\usepackage{graphicx}
\graphicspath{ {img/} }

%package for managing hyperlinks
\usepackage{hyperref}

%package for added blue colored boxes
\usepackage[most]{tcolorbox}

%package that allowed removing indent from enumerate environment
\usepackage{enumitem}

%package that allows for negating nearly any symbol
\usepackage{centernot}

\begin{document}
\large
\maketitle
\noindent This document contains a list of definitions and a list of theorems.
\\\\
Note that the exam will probably require you to PROVE some of these theorems, so you should refer back to the original notes for the proofs.
\\\\
Boxes cover content in more detail. Titles of some theorems are given in italics.
\tableofcontents
\newpage

\section{Definitions}
\paragraph{Arbitrary union}
We can define the arbitrary union
\begin{equation}
\bigcup_{i \in I} X_{i}
\end{equation}
to be the union of all sets $ X_{i} $.
\paragraph{Arbitrary intersection}
We can define the arbitrary intersection
\begin{equation}
\bigcap_{i \in I} X_{i}
\end{equation}
to be the intersection of all sets $ X_{i} $.
\paragraph{Modulus (complex number)}
The modulus of $ z = x + iy $ is defined as:
\begin{equation}
|z| = \sqrt{x^{2} + y^{2}}
\end{equation}
\paragraph{Argument (complex number)}
The argument of $ z = x + iy $ is defined as:
\begin{equation}
\arg(z) = \arctan(\frac{y}{x})
\end{equation}
\paragraph{Fundamental Theorem of Algebra}
If a polynomial is of degree $ n $ then it has $ n $ roots in the complex plane.
\paragraph{Root of unity}
Any $ z $ that is a solution to $ z^{n} = 1 $.
\paragraph{Supremum}
A real number $ b $ is the sup of a set $ S $ if:
\begin{itemize}
\item every $ s \in S $ is less than $ b $ (i.e. $ b $ is an upper bound)
\item $ b $ is less than every other upper bound of $ S $
\end{itemize}
\paragraph{Infimum}
A real number $ b $ is the inf of a set $ S $ if:
\begin{itemize}
\item every $ s \in S $ is greater than $ b $ (i.e. $ b $ is a lower bound)
\item $ b $ is greater than every other lower bound of $ S $
\end{itemize}
\paragraph{Bounded above}
A set is bounded above if it has an upper bound.
\paragraph{Bounded below}
A set is bounded above if it has a lower bound.
\paragraph{Coprime}
We say two integers $ a $ and $ b $ are coprime if $ \gcd(a, b) = 1 $.
\paragraph{Fundamental Theorem of Arithmetic}
Every integer greater than 1 can be expressed as a unique product of primes.
\paragraph{$ \equiv$ (congruence operator)}
We say $ a \equiv b \mod m $ iff $ (a - b) $ is a multiple of $ m $.
\paragraph{Binary relation}
A binary relation $ \sim $ on a set $ S $ is a mapping satisfying $ S \times S \to \textrm{Bool} $.
\paragraph{Reflexive}
A binary relation $ \sim $ is reflexive if for any $ a \in S $:
\begin{equation}
a \sim a
\end{equation}
\paragraph{Symmetric}
A binary relation $ \sim $ is symmetric if:
\begin{equation}
a \sim b \Leftrightarrow b \sim a
\end{equation}
\paragraph{Transitive}
A binary relation $ \sim $ is transitive if:
\begin{equation}
a \sim b \land b \sim c \Rightarrow a \sim c
\end{equation}
\paragraph{Equivalence relation}
A binary relation $ \sim $ is an equivalence relation iff it is reflexive, symmetric and transitive.
\paragraph{Equivalence class}
The equivalence class of $ b \in S $ is the set of all $ s \in S $ such that $ b \sim s $.
\paragraph{Injective}
A function $ f : A \to B $ is injective if every $ b \in B $ has at MOST one $ a \in A $ that maps to it. Mathematically:
\begin{equation}
\forall x, y \in A \quad f(x) = f(y) \Rightarrow x = y
\end{equation}
\paragraph{Surjective}
A function $ f : A \to B $ is surjective if every $ b \in B $ has at LEAST one $ a \in A $ that maps to it. Mathematically:
\begin{equation}
\forall b \in B \quad \exists a \in A \quad s.t. \quad f(a) = b
\end{equation}
\paragraph{Bijective}
A function $ f : A \to B $ is bijective iff it is both surjective and injective.
\paragraph{(Two-sided) inverse function}
The inverse of a function $ f : A \to B $ is defined as $ f^{-1} : B \to A $, and satisfies:
\begin{align}
f^{-1} \circ f \equiv f \circ f^{-1} \equiv Id_{A}
\end{align}
\paragraph{Power set}
The power set of $ S $ is the set of all subsets of $ S $.
\paragraph{Countably infinite}
An infinitely-sized set $ S $ is countably infinite if there exists a bijection $ f : \Z_{> 0} \to S $.

\section{Theorems}
\subsection{Logic and Sets}
\textit{Contrapositive law}
\\
For any Boolean statements $ P, Q $:
\begin{equation}
(P \Rightarrow Q) \Leftrightarrow (\neg Q \Rightarrow \neg P)
\end{equation}
\\
To negate any logical statements, we can flip the $ \forall $ and $ \exists $ signs and negate the predicate (the mathematical statement at the end).
\subsection{Complex numbers}
Complex numbers can be added, subtracted, multiplied and divided.
\\\\
For any complex number $ z $,
\begin{equation}
z \overline{z} = |z| ^{2}
\end{equation}
\\
For any complex numbers $ z_{1}, z_{2} $:
\begin{align}
\overline{z_{1} + z_{2}} & = \overline{z_{1}} + \overline{z_{2}} \\
\overline{z_{1} - z_{2}} & = \overline{z_{1}} - \overline{z_{2}} \\
\overline{z_{1} z_{2}} & = \overline{z_{1}} \cdot \overline{z_{2}} \\
\overline{(\frac{z_{1}}{z_{2}})} & = \frac{\overline{(z_{1})}}{\overline{(z_{2})}} \\
\overline{(z_{1})^{n}} & = (\overline{z_{1}})^{n}
\end{align}
The inverse of a complex number $ z $ is such that:
\begin{equation}
z z^{-1} = 1
\end{equation}
\\
Any complex number $ z $ can be represented in exponential form:
\begin{equation}
z = r e^{i \theta}
\end{equation}
where $ r $ is the modulus of $ z $ and $ \theta $ is the argument.
\\\\
\textit{Euler's formula}
\begin{equation}
e^{i \theta} = \cos(\theta) + i \sin(\theta)
\end{equation}
This can be proven using power series.
\\\\
\textit{De Moivre's Theorem}
\\
For any complex number $ z $:
\begin{equation}
z^{n} = r^{n} (\cos(n \theta) + i \sin(n \theta))
\end{equation}
This can be proven by induction or using Euler's formula.
\\\\
\textit{Fundamental Theorem of Algebra}
\\
Any polynomial of degree $ n $ has $ n $ roots in the complex plane.
\\\\
\textit{Roots of unity}
\\
Any root of unity (i.e. a solution to $ z^{n} = 1 $) can be expressed in the form:
\begin{equation}
z = e^{\frac{2 \pi k i}{n}}, \quad k \in \Z_{\geq 0}
\end{equation}

\subsection{Induction}
\textit{Weak induction}
\\
If we take $ P $ as a predicate:
\begin{equation}
P(0) \land (P(k) \Rightarrow P(k+1)) \Rightarrow P(n) \quad (\forall n)
\end{equation}
Note that we do not have to start at 0:
\begin{equation}
P(m) \land (P(k) \Rightarrow P(k+1)) \Rightarrow P(n) \quad (\forall n \geq m)
\end{equation}
\\
\textit{Strong induction}
\\
If we take $ P $ as a predicate:
\begin{equation}
P(0) \land ((\forall j \leq k \ P(j)) \Rightarrow P(k+1)) \Rightarrow P(n) \quad (\forall n)
\end{equation}
Note that we do not have to start at 0:
\begin{equation}
P(m) \land ((\forall j \leq k \ P(j)) \Rightarrow P(k+1)) \Rightarrow P(n) \quad (\forall n \geq m)
\end{equation}
\begin{tcolorbox}[breakable]
We must prove $ P(k) \Rightarrow P(k+1) $ for weak induction.
\\
We must prove $ (\forall j \leq k \ P(j)) \Rightarrow P(k+1) $ for strong induction.
\\\\
Strong induction is mathematically equivalent to weak induction.
\end{tcolorbox}

\subsection{Number Theory}
\textit{Completeness Axiom}
\\
A set $ S $ has a least upper bound (sup) iff:
\begin{itemize}
\item $ S $ is non-empty
\item $ S $ is bounded above
\end{itemize}
Corresponding statement for greatest lower bound (inf).
\\\\
Every real number has a decimal expansion.
\\\\
A number is rational $ \Leftrightarrow $ it has a periodic decimal expansion.
\\\\
For any $ x \in \Z_{\geq 0} $, $ (x \ | \ a) \textrm{ and } (x \ | \ b) \Rightarrow (x \ | \ \gcd(a, b)) $.
\\
($ x \ | \ a  $ means "x divides a")
\\\\
If $  (x \ | \ ab) $ but $ \gcd(a, x) = 1 $ then $ (x \ | \ b ) $.
\\
\begin{tcolorbox}[breakable]
N.B. In general, $ \gcd(a, b) = 1 $ doesn't mean that $ (a \ \centernot| \ b) $.
\end{tcolorbox}
\noindent
\\
\textit{Euclid's Algorithm}
\\
To find the $ gcd $ of $ a $ and $ b $ where $ a > b $. Write:
\begin{align}
a & = q_{1} b + r_{1} \\
b & = q_{2} r_{1} + r_{2} \\
r_{1} & = q_{3} r_{2} + r_{3} \\
& \vdots \\
r_{n-1} & = q_{n} r_{n-1} + r_{n}
\end{align}
Continue until $ r_{n} = 0 $, return $ r_{n-1} $.
\\
\begin{tcolorbox}[breakable]
In Haskell notation, Euclid's algorithm is specified by:
\begin{verbatim}
gcd(a, 0) = 0
gcd(a, b) = gcd(b, a mod b)
\end{verbatim}
\end{tcolorbox}
\noindent
\\
\textit{Bezout's Theorem}
\begin{equation} \label{bezout}
\gcd(a, b) = \lambda a + \mu b \quad (\textrm{for some } \lambda, \mu \in \R)
\end{equation}
\begin{tcolorbox}[breakable]
We can write $ a $ and $ b $ as:
\begin{align}
a & = \alpha \gcd(a, b) \\
b & = \beta \gcd(a, b)
\end{align}
In general the solution to equation \ref{bezout} is given by:
\begin{equation}
\gcd(a, b) = (\lambda + \beta n) a + (\mu - \alpha n) b
\end{equation}
noting that the extra terms will always cancel out. So we have a set of solutions ($ \lambda_{n}, \mu_{n} $), where:
\begin{equation}
\lambda_{n} = \lambda + \beta n, \quad \mu_{n} = \mu - \alpha n
\end{equation}
\end{tcolorbox}
\noindent
\\
Every integer larger than $ 1 $ can be written as a product of primes.
\\
(use strong induction to prove)
\\\\
\textit{Fundamental Theorem of Arithmetic (proof not needed)}
\\
Every integer larger than $ 1 $ can be written UNIQUELY as a product of primes.

\subsection{Modular arithmetic}
$ (\equiv \mod m) $ is an equivalence relation.
\\\\
If $ a \equiv b \mod m $ and $ c \equiv d \mod m $:
\begin{align}
a + c & \equiv b + d \mod m \\
a - c & \equiv b - d \mod m \\
ac & \equiv bd \mod m
\end{align}
If $ a \equiv b \mod m $, then for any $ n $ :
\begin{equation}
a^{n} \equiv b^{n} \mod m
\end{equation}
\\
\textit{Fermat's Little Theorem}
\\
For any prime $ p $ and $ a \in \Z $:
\begin{equation}
a^{p} \equiv a \mod p
\end{equation}

\subsection{Equivalence relations and functions}
A binary relation $ \sim $ is an equivalence relation iff it is reflexive, symmetric and transitive.
\\\\
The equivalence class of $ b \in S $ is the set of all $ s \in S $ such that $ b \sim s $.
\\\\
The set of equivalence classes of a set $ S $ form a partition of $ S $. In other words, each equivalence class is a disjoint subset of $ S $.
\\\\
A function has an inverse iff it is a bijection.
\\\\
If $ f $ and $ g $ are injective, so is $ g \circ f $.
\\
If $ f $ and $ g $ are surjective, so is $ g \circ f $.
\\
If $ f $ and $ g $ are bijective, so is $ g \circ f $.
\\\\
If we define $ f \sim g $ as "there exists a bijection between $ f $ and $ g $", then $ \sim $ is an equivalence relation.
\\\\
Let $ P(A) $ denote the power set of a set $ A $.
\\
There exists no bijection $ A \to P(A) $.
\\\\
\textit{Examples of countably finite/infinite sets}
\\
$ \Z, \Q $ are countably infinite.
\\
$ \R, \C , \R \setminus \Q $ (set of all irrationals) are uncountably infinite.

\subsection{Combinatorics}
\textit{Multiplicative principle}
\\
In any $ n $-stage process, where the $ r^{th} $ stage can be completed in $ a_{r} $ ways, the total number of was of completing the process is:
\begin{equation}
\prod_{r = 0}^{n} a_{r}
\end{equation}
i.e. just multiply the number of options together.
\\\\
\\
\textit{Multinomials}
\\
When expanding $ (a + b + c + ...)^{n} $, with $ r $ variables in the brackets, the general term of any coefficient is $ a^{\alpha} b^{\beta} c^{\gamma}... $, where $ \alpha + \beta + \gamma + ... = n $.
\\
The coeffient in the expansion for any term $ a^{\alpha} b^{\beta} c^{\gamma}... $ is given by:
\begin{equation}
\begin{pmatrix}
n \\ \alpha,  \beta,  \gamma ...
\end{pmatrix}
=
\frac{n!}{\alpha! \ \beta! \ \gamma! \ ... }
\end{equation}
\\
\begin{tcolorbox}[breakable]
Note if we set $ r = 2 $ in the above equation we get the familiar binomial coefficient:
\begin{align}
\begin{pmatrix}
n \\ \alpha,  \beta
\end{pmatrix}
& =
\frac{n!}{\alpha! \ \beta!} \\
& = \frac{n!}{\alpha! \ (n - \alpha)!}
\end{align}
since $ \beta = n - \alpha $.
\end{tcolorbox}

\end{document}