\documentclass{article}
\title{M1F Summary Notes (JMC Year 1, 2017/2018 syllabus)}
\date{}
\author{Fawaz Shah}

% Packages for adding hyperlinks to table of contents
\usepackage{color}   %May be necessary if you want to color links
\usepackage{hyperref}
\hypersetup{
    colorlinks=true, %set true if you want colored links
    linktoc=all,     %set to all if you want both sections and subsections linked
    linkcolor=black,  %choose some color if you want links to stand out
}

%package that allows aligned equations
\usepackage{amsmath}

%package that allows notation for extra mathematical symbols
\usepackage{amssymb}

%new commands for popular sets for ease of use
\newcommand{\R}{\mathbb{R}}
\newcommand{\N}{\mathbb{N}}
\newcommand{\Z}{\mathbb{Z}}

% renaming command to writing vectors in bold notation
\renewcommand{\vec}[1]{\mathbf{#1}}

%package for managing images
\usepackage{graphicx}
\graphicspath{ {img/} }

%package for managing hyperlinks
\usepackage{hyperref}

%package for added blue colored boxes
\usepackage{tcolorbox}

%package that allowed removing indent from enumerate environment
\usepackage{enumitem}

\begin{document}
\large
\maketitle
\noindent This document contains a list of definitions and a list of theorems.
\\\\
Note that the exam will probably require you to PROVE some of these theorems, so you should refer back to the original notes for the proofs.
\\\\
Boxes cover content in more detail.
\tableofcontents
\newpage

\section{Definitions}
\section{Theorems}
\subsection{Sets}
\subsection{Complex numbers}
\subsection{Number theory}

\textit{Bezout's Theorem}
\begin{equation} \label{bezout}
\gcd(a, b) = \lambda a + \mu b \quad (\textrm{for some } \lambda, \mu \in \R)
\end{equation}
\begin{tcolorbox}
Let:
\begin{itemize}
\item $ a = \alpha \gcd(a, b) $
\item $ b = \beta \gcd(a, b) $
\end{itemize}
In general the solution to equation \ref{bezout} is given by:
\begin{equation}
\gcd(a, b) = (\lambda + \beta n) a + (\mu - \alpha n) b
\end{equation}
noting that the extra terms will always cancel out. So we have a set of solutions ($ \lambda_{n}, \mu_{n} $), where:
\begin{equation}
\lambda_{n} = \lambda + \beta n, \quad \mu_{n} = \mu - \alpha n
\end{equation}
\end{tcolorbox}
\noindent
\\
Every integer $ > 1 $ can be written as a product of primes.
\\
(use strong induction to prove)
\\\\
\textit{Fundamental Theorem of Arithmetic}
\\
Every integer $ > 1 $ can be written UNIQUELY as a product of primes.
\\
(proof not needed)
\subsection{Equivalence relations and functions}
\subsection{Combinatorics}

\end{document}