\documentclass{article}
\title{M1J2 Summary Notes (JMC Year 1, 2017/2018 syllabus)}
\date{}
\author{Fawaz Shah}

% Packages for adding hyperlinks to table of contents
\usepackage{color}   %May be necessary if you want to color links
\usepackage{hyperref}
\hypersetup{
    colorlinks=true, %set true if you want colored links
    linktoc=all,     %set to all if you want both sections and subsections linked
    linkcolor=blue,  %choose some color if you want links to stand out
}

%package that allows aligned equations
\usepackage{amsmath}

%package that allows notation for popular sets (N, R, Q, etc.)
\usepackage{amssymb}

%renaming commands for natural and real number sets for ease of use
\newcommand{\R}{\mathbb{R}}
\newcommand{\N}{\mathbb{N}}


\begin{document}
\large
\maketitle
\begin{center}
STILL UNDER CONSTRUCTION
\end{center}
\noindent Dr Lawn refers to propositions, theorems, corollaries and lemmas. In this document I will refer to them all as 'theorems'.
\\\\
\noindent This document only contains a list of definitions and a list of theorems.
\tableofcontents
\newpage
\part{Abstract Linear Algebra}

\section{Definitions}
\paragraph{Vector space}

\section{Theorems}

\newpage
\part{Group Theory}

\section{Definitions}

\section{Theorems}

\newpage
\part{Analysis}

\section{Definitions}
\paragraph{Sequence}
A sequence is simply a map $ f: \N \to \R $, denoted by $ a_{n} $
\paragraph{Convergence (as $ n \to \infty $)}
A sequence $ a_{n} $ converges to a limit L if for all real numbers $ \epsilon > 0 $, there exists an $ N \in \N $ such that for all $ n > N $ we have $ |a_{n} - L| < \epsilon $.
\begin{equation}
\forall \epsilon > 0 \quad \exists N \in \N \quad s.t \quad \forall n > N \quad |a_{n} - L| < \epsilon
\end{equation}
\paragraph{Tends to infinity (sequence)}
We say a sequence tends to infinity if for all $ R \in \R $, the sequence $ a_{n} $ is eventually bigger than $ R $.
\begin{equation}
\forall R \in \R \quad \exists N \in \N \quad s.t. \quad \forall n > N \quad a_{n} > R
\end{equation}
\paragraph{Shift}
The shift of a sequence by say, k, is the sequence $ b_{n} = a_{n + k} $
\paragraph{Triangle inequality}
The general triangle inequality is:
\begin{equation}
|x - y| < |x - z| + |z - y|
\end{equation}
Setting $ z = 0 $ gives us:
\begin{align}
|x - y| & > |x| - |y|
\end{align}
Then setting $ y = - y $ gives us the familiar case:
\begin{align}
|x + y| & < |x| + |y|
\end{align}
\paragraph{Bounded above}
A sequence $ a_{n} $ is bounded above if there's a real number $ A $ such that $ a_{n} < A $ for all $ n $.
\paragraph{Bounded below}
A sequence $ a_{n} $ is bounded below if there's a real number $ A $ such that $ a_{n} > A $ for all $ n $.
\paragraph{Bounded}
A sequence $ a_{n} $ is bounded if there's a real number $ A $ such that $ |a_{n}| < A $ for all $ n $.
\paragraph{Increasing}
A sequence is increasing if $ a_{n + 1} \geq a_{n} $ for all n.
\paragraph{Strictly increasing}
A sequence is strictly increasing if $ a_{n + 1} > a_{n} $ for all n.
\paragraph{Decreasing}
A sequence is decreasing if $ a_{n + 1} \leq a_{n} $ for all n.
\paragraph{Strictly decreasing}
A sequence is strictly decreasing if $ a_{n + 1} < a_{n} $ for all n.
\paragraph{Monotonic}
A sequence is monotonic if it is increasing or decreasing.
\paragraph{Supremum}
The supremum A of a set $ S $ is the least upper bound of that set i.e. the smallest number such that $ \forall s \in S \quad s \leq A $
\paragraph{Infimum}
The infimum B of a set $ S $ is the greatest lower bound of that set i.e. the largest number such that $ \forall s \in S \quad s \geq B $
\paragraph{Subsequence}
A subsequence of $ a_{n} $ is a sequence $ a_{f(n)} $, where $ f(n) $ is a strictly increasing function.
\paragraph{Cauchy sequence}
A sequence is Cauchy if the terms get arbitrarily close to one another. To put it mathematically:
\begin{equation}
\forall \epsilon > 0 \quad \exists N \in \N \quad s.t \quad \forall m,n \geq N \quad |a_{n} - a_{m}| < \epsilon
\end{equation}
\paragraph{Partial sum}
The $ n^{th} $ partial sum $ S_{n} $ of a sequence $ a_{n} $ is the sum of terms up to that point:
\begin{equation}
S_{n} = \sum_{i=1}^{n} a_{n}
\end{equation}
\paragraph{Summable}
A sequence is summable if the sequence of its partial sums converges. The limit of the sequence of partial sums will be:
\begin{equation}
L = \sum_{i=1}^{\infty} a_{n}
\end{equation}
\paragraph{Absolutely summable}
A sequence $ a_{n} $ is absolutely summable if $ |a_{n}| $ is summable.
\paragraph{Conditionally summable}
A sequence is conditionally summable if it is summable but not absolutely summable.
\paragraph{Power series}
The power series associated with a sequence $ a_{n} $ is the sequence of partial sums:
\begin{equation}
\sum_{i=1}^{n} a_{i}x^{i}
\end{equation}
\paragraph{Radius of convergence}
The radius of convergence R of a power series $ P(x) $ is defined as the largest $ x $ for which $ P(x) $ is convergent.
\paragraph{Limit as $ x \to \infty $ (function)}
A function $ f(x) $ tends to a limit $ L $ as $ x \to \infty $ if for all real numbers $ \epsilon > 0 $, there exists an $ R \in \R $ such that for all $ x \geq R $ we have $ |f(x) - L| < \epsilon $.
\begin{equation}
\forall \epsilon > 0 \quad \exists R \in \R \quad s.t \quad \forall x > R \quad |f(x) - L| < \epsilon
\end{equation}
\paragraph{Tends to infinity (function)}
A function $ f(x) $ tends to infinity as $ x \to \infty $ if for any $ M \in \R $ there exists an $ R \in \R $ such that if $ x > M $ then $ f(x) > R $.
\begin{equation}
\forall M \in \R \quad \exists R \in \R \quad s.t. \quad x > M \implies f(x) > R
\end{equation}
\paragraph{One-sided limit (function}
A function $ f(x) $ tends to a limit $ L $ as $ x \to a^{-} $ if for any $ \epsilon > 0 $ there exists a $ \delta > 0 $ such that if $ x \in (a, a - \delta) $ then $ |f(x) - L| < \epsilon $
\\\\
Same format for the other sided limit ($ x \to a^{+} $)
\\
(Note that $ \epsilon - \delta $ definition is only used for limits as x tends to a finite number a, not infinity)
\paragraph{Continuous (simple def.)}
\paragraph{Continuous (complicated def.)}

\section{Theorems}

\end{document}