%using document class from KOMA-Script
\documentclass{scrartcl}
\title{M1M1 Summary Notes}
\subtitle{JMC Year 1, 2017/2018 syllabus}
\date{}
\author{Fawaz Shah}

% Packages for adding hyperlinks to table of contents
\usepackage{color}   %May be necessary if you want to color links
\usepackage{hyperref}
\hypersetup{
    colorlinks=true, %set true if you want colored links
    linktoc=all,     %set to all if you want both sections and subsections linked
    linkcolor=black,  %choose some color if you want links to stand out
}

%package that allows aligned equations
\usepackage{amsmath}

%package that allows notation for extra mathematical symbols
\usepackage{amssymb}

%new commands for popular sets for ease of use
\newcommand{\R}{\mathbb{R}}
\newcommand{\N}{\mathbb{N}}
\newcommand{\Z}{\mathbb{Z}}
\newcommand{\Q}{\mathbb{Q}}
\newcommand{\C}{\mathbb{C}}

% renaming command to writing vectors in bold notation
\renewcommand{\vec}[1]{\mathbf{#1}}

%package for managing images
\usepackage{graphicx}
\graphicspath{ {../img/} }

%package for managing hyperlinks
\usepackage{hyperref}

%package for added blue colored boxes
\usepackage[most]{tcolorbox}

%package that allowed removing indent from enumerate environment
\usepackage{enumitem}

%package that allows for negating nearly any symbol
\usepackage{centernot}

\begin{document}
\large
\maketitle
\begin{center}
UNDER CONSTRUCTION
\end{center}
\noindent This document contains a bunch of definitions and techniques, sorted by category.
\\\\
M1M1 is more about applying mathematical methods rather than proving theorems, so sly manipulation of mathematical techniques is crucial.
\\\\
Boxes cover content in more detail.
\tableofcontents
\newpage

\subsection{Fundamental Theorem of Calculus}
The fundamental theorem of calculus defines the antiderivative $ F(x) $ and shows the relation between differentiation and integration. It can be expressed in several different forms:
\begin{equation} \label{ftcalculus1}
\frac{d}{dx} \int_{a}^{x} f(t) \ dt = f(x)
\end{equation}
\begin{equation}
\int_{a}^{b} f(x) \ dx = F(b) - F(a)
\end{equation}
Note that in equation \ref{ftcalculus1}, any lower bound $ a $ and dummy variable $ t $ can be picked, without affecting the validity of the theorem.

\end{document}